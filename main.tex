\documentclass[11pt]{beamer}
\usepackage[utf8]{inputenc}
\usepackage[spanish]{babel}
\usepackage{amsmath}
\usepackage{amsfonts}
\usepackage{amssymb}
\usepackage{graphicx}
\usepackage{lipsum}
\usepackage{ragged2e}
\usepackage{hyperref}
\usepackage{float}
\usepackage{url}
\usetheme{Madrid}
\newcommand{\celda}[1]{
	\begin{minipage}{2.5cm}
		\vspace{5mm}
		#1
		\vspace{5mm}
	\end{minipage}
}

\author[Jessica]{Jessica Mariana Carrasco Ceballos\inst{1} \& Docentes-investigadores: Dr.Gonzalo Aranda - Dr. Adolfo Centeno  \inst{1}}
\title[Docking molecular]{Docking molecular de la proteína E del SARS-CoV-2 con la amantadina como ligando}
\date{8 de enero de 2020} 
\subtitle{Acoplamiento molecular con AutoDock Tools y AutoDock Vina}
\institute[UV]{
	\inst{1}
		Universidad Veracruzana. \\Instituto de Investigaciones Cerebrales.\\
		\vspace{2mm}
	
}

\AtBeginSection[]
{
	\begin{frame}<beamer>{Contenido}
		\tableofcontents[currentsection,currentsubsection]
	\end{frame}
}


\begin{document}
	
	\begin{frame}
		\maketitle
	\end{frame}

	\begin{frame}{Contenido}
		\tableofcontents
	\end{frame}

	\section{Resumen}
		\begin{frame}{Resumen}
			\justifying El SARS-CoV-2 es un tipo de virus causante de la enfermedad por coronavirus 2019, el cual provocó una pandemia mundial. Su transmisión se lleva a cabo por contacto con las secreciones de las personas infectadas, dentro de sus signos y símtomas se encuentran fiebre, estornudos, dolor de garganta e insuficiencia respiratoria.
		\end{frame}
	
	\section{Introducción}
		\begin{frame}{Introducción}
			\justifying El nuevo coronavirus SARS-CoV-2 se descubrió en Wuhan, China, en diciembre del 2019. 
			
			Son virus esféricos, con un diámetro aproximado de 80-120 nm. Su superficie esta constituida por trímeros de la glicoproteína viral S (Spike), la cual media la entrada del virus a la célula huésped, la envoltura viral se encuentra reforzada por la glicoproteína de membrana (M) y por la proteína de envoltura (E).
			
			La proteína E está constituida por 75 aminoácidos, y en diversas investigaciones se ha sugerido que la carencia de esta proteína disminuye el daño en ratones que han sido infectados por COVID-19.
			\end{frame}
			
			\begin{frame}{Introducción}
			Su transmisión es a través del contacto con las secreciones de las personas infectadas.
			
			Las manifestaciones clínicas más comúnes son: fiebre, tos seca, neumonía leve, fatiga, disnea, disminución de la saturación de oxígeno, taquipnea, síndrome de dificultad respiratoria aguda (SDRA), entre otras.
			
			Actualmente se está trabajando para encontrar un fármaco que pueda ayudar a combatir esta enfermedad, por lo que Aranda y colaboradores han propuesto el uso del antiviral amantadina.
		\end{frame}
		
	
	\section{Docking molecular}
		\begin{frame}{Docking molecular}
			\justifying El Docking o acoplamiento molecular es un método utilizado que permite encontrar compuestos novedosos de interés terapéutico, a través de la predicción de las interacciones de dos determinadas moléculas.
			
			Se realizó de la siguiente forma: 1) Se utilizó AutoDockTools y AutoDock Vina, la proteína E del SARS-CoV-2, amantadina (DB00915) como ligando y los aminoácidos LEU18, LEU 19 y ASN15. Antes de iniciar se preparó la proteína.
		\end{frame}
	
	\begin{frame}{Docking molecular}
			\justifying 2) Se sube la proteína al programa y se visualiza en cintas. Después se integra el ligando amantadina (DB00915) en formato .pdb.
			
			3) Posteriormente se prepara la macromolécula, se eligen los aminoácidos con los que se va a trabajar y mediante Gird Box se configura el espacio de búsqueda, así se establecerá la ubicación y tamaño del área 3D que será analizada. 
			
			4) Se utilizó un CPU de 4, una semilla aleatoria de 1020676528. Finalmente, se realizó el Docking que mostró las interacciones.
		\end{frame}
		
		\begin{frame}{Docking molecular.}
			\justifying
			A continuación se muestran algunas imágenes que se obtuvieron durante el proceso de alineamiento molecular:
			\begin{figure}[H]
				\centering
				\includegraphics[scale=0.9]{cintas.png}
				\caption{Proteína E del SARS-CoV-2}
				\label{fig: Figura1}
			\end{figure}
		\end{frame} 
		
			\begin{frame}{Docking molecular.}
			\justifying
			\begin{figure}[H]
				\centering
				\includegraphics[scale=0.9]{ligando.png}
				\caption{Amantadina (DB00915) como ligando}
				\label{fig: Figura2}
			\end{figure}
		\end{frame} 
		
		\begin{frame}{Docking molecular.}
			\justifying
			\begin{figure}[H]
				\centering
				\includegraphics[scale=0.9]{caja.png}
				\caption{Extensión y ubicación del área tridimensional (3D)}
				\label{fig: Figura3}
			\end{figure}
		\end{frame} 
		
		\begin{frame}{Docking molecular.}
			\justifying
			\begin{figure}[H]
				\centering
				\includegraphics[scale=0.9]{semilla.png}
				\caption{Docking: semilla aleatoria}
				\label{fig: Figura4}
			\end{figure}
		\end{frame}
		
		\begin{frame}{Docking molecular.}
			\justifying
			\begin{figure}[H]
				\centering
				\includegraphics[scale=0.9]{docking.png}
				\caption{Docking mostrando las interacciones}
				\label{fig: Figura5}
			\end{figure}
		\end{frame}
	
	\section{Conclusión}
		\begin{frame}{Conclusión}
			\justifying En conclusión el acoplamiento molecular o Docking, es un método sumamente útil para conocer las diversas interacciones que tiene una determinada molécula o proteína con múltiples ligandos. Lo anterior es importante, ya que nos permite saber si algún ligando farmacológico genera alguna respuesta una vez que se ha unido con la proteína, y así poder descubrir más usos de estos fármacos en una enfermedad específica, como en este caso, el coronavirus.
		\end{frame}
	
	

%\appendix
\section{Referencias}
%\subsection<presentation>*{Referencias}

\begin{frame}{Referencias}

\begin{thebibliography}{5}
	
	\beamertemplatearticlebibitems
	
	\bibitem{Abreu 2020}
	Abreu, G. E. A., Aguilar, M. E. H., Covarrubias, D. H., y Durán, F. R.
	\newblock{Amantadine as a drug to mitigate the effects of COVID-19. Medical Hypotheses, 140(April), 1–3. https://doi.org/10.1016/j.mehy.2020.109755}
	
	\bibitem{Aranda 2020}
	Ara, Ramiro;Aranda-Abreu, E
	\newblock{Amantadine Treatment for People with COVID-19. (January), 19–21}
	
	\bibitem{Fehlmann 2005}
	Fehlmann P., E., Le Corre P., N., Abarca V., K., Godoy M., P.
	\newblock{Búsqueda de resistencia a amantadina en cepas de virus influenza A aisladas en Santiago de Chile, entre los años 2001 y 2002. Revista Chilena de Infectología, 22(2), 141–146. https://doi.org/10.4067/s0716- 10182005000200004}

	\bibitem{Harapan 2020}
	Harapan, H., Itoh, N., Yufika, A.
	\newblock{Coronavirus disease 2019 (COVID-19): A literature review. Journal of Infection and Public Health, 13(5), 667–673}
	
\end{thebibliography}
\end{frame}
	
	\begin{frame}{Referencias}

\begin{thebibliography}{5}

    \beamertemplatearticlebibitems
	
	\bibitem{Harrison 2020}
	Harrison, A. G., Lin, T., y Wang, P.
	\newblock{Mechanisms of SARS-CoV-2 Transmission and Pathogenesis. Trends in Immunology, 41(12), 1100–1115. https://doi.org/10.1016/j.it.2020.10.004}
	
	\bibitem{Huraimel 2020}
	Huraimel, K. Al, Alhosani, M., Kunhabdulla, S., y Stietiya, M. H.
	\newblock{SARS-CoV-2 in the environment: Modes of transmission, early detection and potential role of pollutions. (January)}
	
	\bibitem{Maté 2017}
	Maté, C. F.
	\newblock{Modelado Molecular Como Herramienta Para El Descubrimiento De Nuevos Fármacos Que Interaccionan Con Proteínas. 21. Retrieved from http://147.96.70.122/Web/TFG/TFG/Memoria/CRISTINA FONT MATE.pdf}
	
\end{thebibliography}
\end{frame}

\begin{frame}{Referencias}

\begin{thebibliography}{5}
	
	\beamertemplatearticlebibitems

    \bibitem{Michael 2021}
	Michael J. Smart, R. B. N. O. C.
	\newblock{COVID-19, una emergencia de salud pública mundial. Revista Clinica Espanola, 55–61}

    \bibitem{Naserghandi 2020}
	Naserghandi, A., Allameh, S. F., y Saffarpour, R.
	\newblock{All about COVID-19 in brief. New Microbes and New Infections, 35, 100678. https://doi.org/10.1016/j.nmni.2020.100678}
	
	\bibitem{Pinzi 2019}
	Pinzi, L., y Rastelli, G.
	\newblock{Molecular docking: Shifting paradigms in drug discovery. International Journal of Molecular Sciences, 20(18). https://doi.org/10.3390/ijms20184331}


\end{thebibliography}
\end{frame}
\end{document}